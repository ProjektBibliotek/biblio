\documentclass[a4paper]{article}
\usepackage{polski}
\usepackage[cp1250]{inputenc}
\usepackage{url}

\title{\bf{Aplikacja webowa biblioteki wykonana przy użyciu technologii PHP i MYSQL}}
\author{{\em Łukasz Wojtas}\\{\em Patrycja Pieniążek}\\{\em Karolina Pasiut}\\{\em Mateusz Zygmunt}\\{\em Grzegorz Wygoda }}

\date{}

\begin{document}

\begin{titlepage}
\maketitle
\thispagestyle{empty}
\bigskip
\begin{center}
Zespołowe przedsięwzięcie inżynierskie\\[2mm]

Informatyka\\[2mm]

Rok. akad. 2017/2018, sem. I\\[2mm]

Prowadzący: dr hab. Marcin Mazur
\end{center}
\end{titlepage}

\tableofcontents
\thispagestyle{empty}

\newpage

\section{Opis projektu}

\subsection{Członkowie zespołu}

\begin{enumerate}
\item Łukasz Wojtas (kierownik projektu),
\item Patrycja Pieniążek,
\item Karolina Pasiut,
\item Mateusz Zygmunt,
\item Grzegorz Wygoda
\end{enumerate}

\subsection{Cel projektu (produkt)}

Celem projektu jest stworzenie aplikacji webowej biblioteki wykonanej przy użyciu technologii PHP i MYSQL. Ma ona na celu skomputeryzowanie, a tym samym usprawnienie obsługi księgozbioru biblioteki. Aplikacja będzie pozwalała na wyszukiwanie pozycji oraz wypożyczenie wybranej przez nas książki.

\subsection{Potencjalny odbiorca produktu (klient)}

Potencjalnym odbiorcą są biblioteki gminne które przeprowadzają informatyzacje i restrukturyzacje zbiorów bibliotecznych.

\subsection{Metodyka}

Projekt będzie realizowany przy użyciu (zaadaptowanej do istniejących warunków) metodyki {\em Scrum}. 

\section{Wymagania użytkownika}

\subsection{User story 1}
Jako czytelnik chciałbym, aby pojawiła się wyszukiwarka pozwalająca szybko znaleźć książkę która mnie interesuje, tak by nie musieć przeglądać całej listy książek.

\subsection{User story 2}
Jako czytelnik chciałabym mieć możliwość wyszukania wszystkich książek danego autora po wpisaniu w wyszukiwarce jego  nazwiska, aby skrócić proces wyszukiwania.

\subsection{User story 3}
Jako czytelnik chciałbym mieć możliwość zalogowania się na swoje indywidualne konto, by móc korzystać z wszystkich usług platformy.

\subsection{User story 4}
Jako czytelnik chciałbym mieć możliwość wyszukiwania książki po jej gatunku, aby nie trzeba było przeglądać książek które mnie nie interesują.

\subsection{User story 5}
Jako czytelnik chciałabym aby strona działała na urządzeniach mobilnych, aby móc wypożyczyć książkę bez używania komputera/laptopa (strona skalowalna do wyświetlacza urządzenia mobilnego).

\subsection{User story 6}
Jako czytelnik chciałabym mieć możliwość sprawdzenia danych kontaktowych biblioteki, aby w razie wystąpienia problemu móc szybko skontaktować się z pracownikami. 

\subsection{User story 7}
Jako czytelnik chciałbym mieć możliwość wypożyczania(rezerwacji) książki online, abym nie musiał stać w kolejce i tracić czasu. 

\subsection{User story 8}
Jako czytelnik chciałbym, aby na stronie internetowej znalazło się komfortowe menu nawigacyjne, aby ułatwić przeglądanie strony i z łatwością móc przechodzić między zakładkami.

\subsection{User story 9}
Jako czytelnik chciałbym aby na stronie można było zobaczyć listę książek najczęściej wypożyczanych przez innych czytelników, aby znaleźć ciekawe, nieznane wcześniej pozycje literackie. 

\subsection{User story 10}
Jako czytelnik chciałbym mieć możliwość zaproponowania książki której biblioteka nie posiada, aby jak najszybciej mieć możliwość przeczytania interesujących tytułów bez konieczności zakupu.(opcjonalnie)

\subsection{User story 11}
Jako czytelnik chciałbym móc sprawdzić termin oddania książki aby oddać lub przedłużyć książkę w terminie, aby uniknąć płacenia kary.

\subsection{User story 12}
Jako czytelnik chciałbym mieć możliwość przedłużenia terminu wypożyczenia książki, aby w razie potrzeby moc zatrzymać książkę na dłużej bez potrzeby pójścia do biblioteki stacjonarnej.

\subsection{User story 13}
Jako czytelnik chciałbym mieć możliwość zarezerwowania (zapisania się w kolejce) książki na stronie internetowej, aby jak najszybciej otrzymać możliwość wypożyczenia książki cieszącej się dużą popularnością w bibliotece.

\subsection{User story 14}
Jako czytelnik chciałbym móc przeczytać krótki opis książki którą chciałbym wypożyczyć, aby wiedzieć o czym ona jest i czy przypadnie mi do gustu.(opcjonalnie)

\subsection{User story 15}
Jako administrator chciałbym mieć dostęp do bazy kont użytkowników, aby móc zarządzać kontami użytkowników: dodawać, usuwać oraz edytować dane.

\subsection{User story 16}
Jako administrator chce mieć pełny dostęp do bazy danych książek, aby móc zarządzać listą dostępnych pozycji: dodawać, usuwać, poprawiać dane książek.

\subsection{User story 17}
Jako pracownik chciałbym możliwość dodawania użytkowników, aby móc obsłużyć nowych klientów.

\subsection{User story 18}
Jako pracownik chciałbym mieć podgląd niewypożyczonych książek, aby mieć zawsze aktualna listę do stacjonarnej obsługi klienta.



\section{Harmonogram}

\subsection{Rejestr zadań (Product Backlog)}

\begin{itemize}
\item Data rozpoczęcia: 25.10.2017.
\item  Data zakończenia: 31.10.2017.
\end{itemize}

\subsection{Sprint 1}

\begin{itemize}
\item Data rozpoczęcia: 07.11.2017.
\item Data zakończenia: 21.11.2017.
\item Scrum Master: Łukasz Wojtas.
\item Product Owner: Karolina Pasiut.
\item Development Team: Patrycja Pieniążek, Grzegorz Wygoda, Mateusz Zygmunt.
\end{itemize}

\subsection{Sprint 2}

\begin{itemize}
\item Data rozpoczęcia: 21.11.2017.
\item  Data zakończenia: 19.12.2017.
\item Scrum Master: Patrycja Pieniążek.
\item Product Owner: Grzegorz Wygoda.
\item Development Team: Karolina Pasiut, Łukasz Wojtas, Mateusz Zygmunt.
\end{itemize}

\subsection{Sprint 3}

\begin{itemize}
\item Data rozpoczęcia: 19.12.2017.
\item  Data zakończenia: 16.01.2018.
\item Scrum Master: Mateusz Zygmunt.
\item Product Owner: Łukasz Wojtas.
\item Development Team: Karolina Pasiut, Patrycja Pieniążek, Grzegorz Wygoda.
\end{itemize}


\section{Product Backlog}

\subsection{Backlog Item 1}
\paragraph{Tytuł zadania.} Zaprojektowanie bazy danych.
\paragraph{Opis zadania.} Zadanie ma na celu stworzenie projektu bazy opartej na technologii MySQL zoptymalizowanej pod kątem działania biblioteki.
\paragraph{Priorytet.} 1.
\paragraph{Definition of Done.} Gotowy do zaimplementowania projekt bazy danych.

\subsection{Backlog Item 2}
\paragraph{Tytuł zadania.} Utworzenie bazy danych.
\paragraph{Opis zadania.} Zadanie ma na celu stworzenie bazy danych, która będzie przechowywała informacje aplikacji zarządzania biblioteką.
\paragraph{Priorytet.} 1.
\paragraph{Definition of Done.} Baza danych gotowa do wprowadzania danych aplikacji.

\subsection{Backlog Item 3}
\paragraph{Tytuł zadania.} Utworzenie szablonu interfejsu webowego.
\paragraph{Opis zadania.} Zadanie ma na celu stworzenie szablonu interfejsu webowego aplikacji pozwalającgo na interakcję klienta z aplikacją.
\paragraph{Priorytet.} 2.
\paragraph{Definition of Done.}Szablon interfejsu webowego obsługujący aplikacje.

\subsection{Backlog Item 4}
\paragraph{Tytuł zadania.} Utworzenie podstron: dodającej, modyfikującej jak i usuwającej wpisy książkowe w bazie danych.
\paragraph{Opis zadania.} Zadanie ma na celu stworzenie podstron, które umożliwią modyfikacje wpisów w bazie danych książek.
\paragraph{Priorytet.} 3
\paragraph{Definition of Done.} Podstrony: modyfikacji, dodawania oraz usuwania wpisów książkowych.

\subsection{Backlog Item 5}
\paragraph{Tytuł zadania.} Utworzenie podstrony modyfikującej, dodającej oraz usuwającej użytkowników zawartych w bazie danych.
\paragraph{Opis zadania.} Zadanie ma na celu stworzenie podstrony, pozwalającej na modyfikowanie bazy użytkowników.
\paragraph{Priorytet.} 3
\paragraph{Definition of Done.} Podstrona modyfikacji, dodawania oraz usuwania bazy użytkowników.

\subsection{Backlog Item 6}
\paragraph{Tytuł zadania.} Utworzenie podstrony wyświetlającej katalog ksiązek.
\paragraph{Opis zadania.} Zadanie ma na celu stworzenie podstrony umożliwiającej podgląd listy ksiązek.
\paragraph{Priorytet.} 5
\paragraph{Definition of Done.} Podstrona wyświetlająca listę książek.

\subsection{Backlog Item 7} 
\paragraph{Tytuł zadania.} Utworzenie skryptów PHP: Modyfikacja, dodawanie i usuwanie książek.
\paragraph{Opis zadania.} Zadanie ma na celu utworzenie skryptów w języku PHP pozwalających na modyfikację danymi książek.
\paragraph{Priorytet.} 2
\paragraph{Definition of Done.} Prawidłowe działanie skryptów pozwalających na modyfikowanie listy ksiązek.

\subsection{Backlog Item 8}
\paragraph{Tytuł zadania.} Utworzenie skryptów PHP: Modyfikacja, dodawanie i usuwanie użytkowników.
\paragraph{Opis zadania.} Zadanie ma na celu utworzenie skryptów w języku PHP pozwalających na modyfikację danymi użytkowników.
\paragraph{Priorytet.} 2
\paragraph{Definition of Done.} Działające skrypty PHP pozwalające na modyfikacje listy użytkowników.

\subsection{Backlog Item 9}
\paragraph{Tytuł zadania.} Utworzenie skryptów PHP: Sprawdzanie wypożyczonych książek przez klienta, prolongata wypozyczonych pozycji.
\paragraph{Opis zadania.} Zadanie ma na celu utworzenie skryptów w języku PHP pozwalających podgląd danych wypożyczonych książek oraz prolongatę wybranych przez klienta pozycji.
\paragraph{Priorytet.} 3
\paragraph{Definition of Done.} Działające skrypty PHP wyświetlające wypożyczone książki oraz obsługujące funkcję prolongaty przez klienta.

\subsection{Backlog Item 10}
\paragraph{Tytuł zadania.} Utworzenie skryptów obługujących logowanie do systemu.
\paragraph{Opis zadania.} Zadanie ma na celu utworzenie skryptów pozwalających zalogować się zarówno klientowi jak i pracownikowi biblioteki do systemu obsługującego bibliotekę.
\paragraph{Priorytet.} 3
\paragraph{Definition of Done.} Działające skrypty umożliwiające logowanie do systemu.

\section{Sprint 1}
\subsection{Cel} <<Określić, w jakim celu tworzony jest przyrost produktu>>.
\subsection{Sprint Planning/Backlog}

\paragraph{Tytuł zadania.} <<Tytuł>>.
\begin{itemize}
\item Estymata: <<szacowana czasochłonność (w ,,koszulkach'')>>.
\end{itemize}

\paragraph{Tytuł zadania.} <<Tytuł>>.
\begin{itemize}
\item Estymata: <<szacowana czasochłonność (w ,,koszulkach'')>>.
\end{itemize}

\paragraph{<<Tutaj dodawać kolejne zadania>>}

\subsection{Realizacja}

\paragraph{Tytuł zadania.} <<Tytuł>>.
\subparagraph{Wykonawca.} <<Wykonawca>>.
\subparagraph{Realizacja.} <<Sprawozdanie z realizacji zadania (w tym ocena zgodności z estymatą). Kod programu (środowisko \texttt{verbatim}): \begin{verbatim}
for (i=1; i<10; i++)
...
\end{verbatim}>>.

\paragraph{Tytuł zadania.} <<Tytuł>>.
\subparagraph{Wykonawca.} <<Wykonawca>>.
\subparagraph{Realizacja.} <<Sprawozdanie z realizacji zadania (w tym ocena zgodności z estymatą). Kod programu (środowisko \texttt{verbatim}): \begin{verbatim}
for (i=1; i<10; i++)
...
\end{verbatim}>>.

\paragraph{<<Tutaj dodawać kolejne zadania>>}


\subsection{Sprint Review/Demo}
<<Sprawozdanie z przeglądu Sprint'u -- czy założony cel (przyrost) został osiągnięty oraz czy wszystkie zaplanowane Backlog Item'y zostały zrealizowane? Demostracja przyrostu produktu>>.

\section{Sprint 2}

\subsection{Cel} <<Określić, w jakim celu tworzony jest przyrost produktu>>.

\subsection{Sprint Planning/Backlog}

\paragraph{Tytuł zadania.} <<Tytuł>>.
\begin{itemize}
\item Estymata: <<szacowana czasochłonność (w ,,koszulkach'')>>.
\end{itemize}

\paragraph{Tytuł zadania.} <<Tytuł>>.
\begin{itemize}
\item Estymata: <<szacowana czasochłonność (w ,,koszulkach'')>>.
\end{itemize}

\paragraph{<<Tutaj dodawać kolejne zadania>>}

\subsection{Realizacja}

\paragraph{Tytuł zadania.} <<Tytuł>>.
\subparagraph{Wykonawca.} <<Wykonawca>>.
\subparagraph{Realizacja.} <<Sprawozdanie z realizacji zadania (w tym ocena zgodności z estymatą). Kod programu (środowisko \texttt{verbatim}): \begin{verbatim}
for (i=1; i<10; i++)
...
\end{verbatim}>>.

\paragraph{Tytuł zadania.} <<Tytuł>>.
\subparagraph{Wykonawca.} <<Wykonawca>>.
\subparagraph{Realizacja.} <<Sprawozdanie z realizacji zadania (w tym ocena zgodności z estymatą). Kod programu (środowisko \texttt{verbatim}): \begin{verbatim}
for (i=1; i<10; i++)
...
\end{verbatim}>>.

\paragraph{<<Tutaj dodawać kolejne zadania>>}


\subsection{Sprint Review/Demo}
<<Sprawozdanie z przeglądu Sprint'u -- czy założony cel (przyrost) został osiągnięty oraz czy wszystkie zaplanowane Backlog Item'y zostały zrealizowane? Demostracja przyrostu produktu>>.

\section*{<<Tutaj dodawać kolejne Sprint'y>>}


\begin{thebibliography}{9}

\bibitem{Cov} S. R. Covey, {\em 7 nawyków skutecznego działania}, Rebis, Poznań, 2007.

\bibitem{Oet} Tobias Oetiker i wsp., Nie za krótkie wprowadzenie do systemu \LaTeX  \ $2_\varepsilon$, \url{ftp://ftp.gust.org.pl/TeX/info/lshort/polish/lshort2e.pdf}

\bibitem{SchSut} K. Schwaber, J. Sutherland, {\em Scrum Guide}, \url{http://www.scrumguides.org/}, 2016.

\bibitem{apr} \url{https://agilepainrelief.com/notesfromatooluser/tag/scrum-by-example}

\bibitem{us} \url{https://www.tutorialspoint.com/scrum/scrum_user_stories.htm}

\end{thebibliography}

\end{document}

