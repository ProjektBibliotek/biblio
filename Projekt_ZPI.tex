\documentclass[a4paper]{article}
\usepackage{polski}
\usepackage[cp1250]{inputenc}
\usepackage{url}

\title{\bf{Aplikacja webowa biblioteki wykonana przy użyciu technologii PHP i MYSQL}}
\author{{\em Łukasz Wojtas}\\{\em Patrycja Pieniążek}\\{\em Karolina Pasiut}\\{\em Mateusz Zygmunt}\\{\em Grzegorz Wygoda }}

\date{}

\begin{document}

\begin{titlepage}
\maketitle
\thispagestyle{empty}
\bigskip
\begin{center}
Zespołowe przedsięwzięcie inżynierskie\\[2mm]

Informatyka\\[2mm]

Rok. akad. 2017/2018, sem. I\\[2mm]

Prowadzący: dr hab. Marcin Mazur
\end{center}
\end{titlepage}

\tableofcontents
\thispagestyle{empty}

\newpage

\section{Opis projektu}

\subsection{Członkowie zespołu}

\begin{enumerate}
\item Łukasz Wojtas (kierownik projektu),
\item Patrycja Pieniążek,
\item Karolina Pasiut,
\item Mateusz Zygmunt,
\item Grzegorz Wygoda
\end{enumerate}

\subsection{Cel projektu (produkt)}

Celem projektu jest stworzenie aplikacji webowej biblioteki wykonanej przy użyciu technologii PHP i MYSQL. Ma ona na celu skomputeryzowanie, a tym samym usprawnienie obsługi księgozbioru biblioteki. Aplikacja będzie pozwalała na wyszukiwanie pozycji oraz wypożyczenie wybranej przez nas książki.

\subsection{Potencjalny odbiorca produktu (klient)}

Potencjalnym odbiorcą są biblioteki gminne, które przeprowadzają informatyzacje i restrukturyzacje zbiorów bibliotecznych.

\subsection{Metodyka}

Projekt będzie realizowany przy użyciu (zaadaptowanej do istniejących warunków) metodyki {\em Scrum}. 

\section{Wymagania użytkownika}

\subsection{User story 1}
Jako czytelnik chciałbym, aby pojawiła się wyszukiwarka pozwalająca szybko znaleźć książkę, która mnie interesuje, tak by nie musieć przeglądać całej listy książek.

\subsection{User story 2}
Jako czytelnik, chciałabym mieć możliwość wyszukania wszystkich książek danego autora po wpisaniu w wyszukiwarce jego nazwiska, aby skrócić proces wyszukiwania.

\subsection{User story 3}
Jako czytelnik, chciałbym mieć możliwość zalogowania się na swoje indywidualne konto, by móc korzystać z wszystkich usług platformy.

\subsection{User story 4}
Jako czytelnik, chciałbym mieć możliwość wyszukiwania książki po jej gatunku, aby nie trzeba było przeglądać książek, które mnie nie interesują.

\subsection{User story 5}
Jako czytelnik, chciałabym aby strona działała na urządzeniach mobilnych, aby móc wypożyczyć książkę bez używania komputera/laptopa (strona skalowalna do wyświetlacza urządzenia mobilnego).

\subsection{User story 6}
Jako czytelnik, chciałabym mieć możliwość sprawdzenia danych kontaktowych biblioteki, aby w razie wystąpienia problemu móc szybko skontaktować się z pracownikami. 

\subsection{User story 7}
Jako czytelnik, chciałbym mieć możliwość wypożyczania(rezerwacji) książki online, abym nie musiał stać w kolejce i tracić czasu. 

\subsection{User story 8}
Jako czytelnik chciałbym, aby na stronie internetowej znalazło się komfortowe menu nawigacyjne, aby ułatwić przeglądanie strony i z łatwością móc przechodzić między zakładkami.

\subsection{User story 9}
Jako czytelnik, chciałbym aby na stronie można było zobaczyć listę książek najczęściej wypożyczanych przez innych czytelników, aby znaleźć ciekawe, nieznane wcześniej pozycje literackie. 

\subsection{User story 10}
Jako czytelnik, chciałbym mieć możliwość zaproponowania książki, której biblioteka nie posiada, aby jak najszybciej mieć możliwość przeczytania interesujących tytułów bez konieczności zakupu.(opcjonalnie)

\subsection{User story 11}
Jako czytelnik, chciałbym móc sprawdzić termin oddania książki aby oddać lub przedłużyć książkę w terminie, aby uniknąć płacenia kary.

\subsection{User story 12}
Jako czytelnik, chciałbym mieć możliwość przedłużenia terminu wypożyczenia książki, aby w razie potrzeby moc zatrzymać książkę na dłużej bez potrzeby pójścia do biblioteki stacjonarnej.

\subsection{User story 13}
Jako czytelnik, chciałbym mieć możliwość zarezerwowania (zapisania się w kolejce) książki na stronie internetowej, aby jak najszybciej otrzymać możliwość wypożyczenia książki cieszącej się dużą popularnością w bibliotece.

\subsection{User story 14}
Jako czytelnik, chciałbym móc przeczytać krótki opis książki, którą chciałbym wypożyczyć, aby wiedzieć o czym ona jest i czy przypadnie mi do gustu.(opcjonalnie)

\subsection{User story 15}
Jako administrator, chciałbym mieć dostęp do bazy kont użytkowników, aby móc zarządzać kontami użytkowników: dodawać, usuwać oraz edytować dane.

\subsection{User story 16}
Jako administrator, chciałbym mieć pełny dostęp do bazy danych książek, aby móc zarządzać listą dostępnych pozycji: dodawać, usuwać, poprawiać dane książek.

\subsection{User story 17}
Jako administrator, chciałbym możliwość dodawania użytkowników, aby móc obsłużyć nowych klientów.

\subsection{User story 18}
Jako administrator, chciałbym mieć podgląd niewypożyczonych książek, aby mieć zawsze aktualna listę do stacjonarnej obsługi klienta.

\subsection{User story 19}
Jako administrator, chciałbym mieć możliwość wypożyczania książek dla czytelników, aby móc obsłużyć klienta.


\section{Harmonogram}

\subsection{Rejestr zadań (Product Backlog)}

\begin{itemize}
\item Data rozpoczęcia: 25.10.2017.
\item  Data zakończenia: 07.11.2017.
\end{itemize}

\subsection{Sprint 1}

\begin{itemize}
\item Data rozpoczęcia: 07.11.2017.
\item Data zakończenia: 21.11.2017.
\item Scrum Master: Łukasz Wojtas.
\item Product Owner: Karolina Pasiut.
\item Development Team: Patrycja Pieniążek, Grzegorz Wygoda, Mateusz Zygmunt.
\end{itemize}

\subsection{Sprint 2}

\begin{itemize}
\item Data rozpoczęcia: 21.11.2017.
\item  Data zakończenia: 19.12.2017.
\item Scrum Master: Patrycja Pieniążek.
\item Product Owner: Grzegorz Wygoda.
\item Development Team: Karolina Pasiut, Łukasz Wojtas, Mateusz Zygmunt.
\end{itemize}

\subsection{Sprint 3}

\begin{itemize}
\item Data rozpoczęcia: 19.12.2017.
\item  Data zakończenia: 16.01.2018.
\item Scrum Master: Mateusz Zygmunt.
\item Product Owner: Łukasz Wojtas.
\item Development Team: Karolina Pasiut, Patrycja Pieniążek, Grzegorz Wygoda.
\end{itemize}


\section{Product Backlog}

\subsection{Backlog Item 1}
\paragraph{Tytuł zadania.} Zaprojektowanie bazy danych.
\paragraph{Opis zadania.} Zadanie ma na celu stworzenie projektu bazy opartej na technologii MySQL zoptymalizowanej pod kątem działania biblioteki z tabelami zawierającymi dane książek, wypożyczeń klientów i użytkowników.
\paragraph{Priorytet.} 1.
\paragraph{Definition of Done.} Gotowy do zaimplementowania projekt bazy danych zawierający informacje podane w opisie.

\subsection{Backlog Item 2}
\paragraph{Tytuł zadania.} Utworzenie bazy danych.
\paragraph{Opis zadania.} Zadanie ma na celu stworzenie bazy danych, która będzie przechowywała informacje aplikacji zarządzania biblioteką  z przykładowymi danymi w każdej z tabel. Baza powstanie na podstawie projektu wykonanego w poprzednim zadaniu.
\paragraph{Priorytet.} 1.
\paragraph{Definition of Done.} Baza danych gotowa do wprowadzania danych aplikacji i testowania działania skryptów.

\subsection{Backlog Item 3}
\paragraph{Tytuł zadania.} Utworzenie szablonu interfejsu webowego.
\paragraph{Opis zadania.} Zadanie ma na celu stworzenie szablonu interfejsu webowego pokazującego graficzny układ strony aplikacji.
\paragraph{Priorytet.} 2.
\paragraph{Definition of Done.}Szablon interfejsu webowego obsługujący aplikację zawierający: logo, menu aplikacji, informacje o bibliotece, puste podstrony aplikacji.

\subsection{Backlog Item 4} 
\paragraph{Tytuł zadania.} Utworzenie skryptów PHP: Modyfikacja, dodawanie i usuwanie książek.
\paragraph{Opis zadania.} Zadanie ma na celu utworzenie skryptów w języku PHP pozwalających na dodawanie, modyfikowanie oraz usuwanie książek z listy.
\paragraph{Priorytet.} 2
\paragraph{Definition of Done.} Prawidłowe działanie skryptów pozwalających na modyfikowanie listy książek, weryfikowane wywołaniem ich przez interfejs webowy i sprawdzeniem efektu działania w bazie MySQL.

\subsection{Backlog Item 5}
\paragraph{Tytuł zadania.} Utworzenie skryptów PHP: Modyfikacja, dodawanie i usuwanie użytkowników.
\paragraph{Opis zadania.} Zadanie ma na celu utworzenie skryptów w języku PHP pozwalających na dodawanie, modyfikowanie oraz usuwanie użytkowników z listy.
\paragraph{Priorytet.} 2
\paragraph{Definition of Done.} Działające skrypty PHP pozwalające na modyfikacje listy użytkowników, weryfikowane wywołaniem ich przez interfejs webowy i sprawdzeniem efektu działania w bazie MySQL.

\subsection{Backlog Item 6}
\paragraph{Tytuł zadania.}Rozbudowa bazy danych na potrzeby obsługi kolejnych funkcji aplikacji.
\paragraph{Opis zadania.}Zadanie ma na celu rozbudowanie bazy danych o kolejne rozszerzenia, umożliwiające dalszą pracę nad projektem.
\paragraph{Priorytet.}2
\paragraph{Definition of Done.}Baza dostosowana do nowych funkcji aplikacji. Dodanie kolejnych tabeli, widoków oraz indeksów.

\subsection{Backlog Item 7}
\paragraph{Tytuł zadania.} Utworzenie podstron: dodającej, modyfikującej jak i usuwającej wpisy książkowe w bazie danych.
\paragraph{Opis zadania.} Zadanie ma na celu stworzenie podstron, które umożliwią modyfikacje wpisów w bazie danych książek.
\paragraph{Priorytet.} 3
\paragraph{Definition of Done.} Działające podstrony aplikacji umożliwiające dodawanie, modyfikowanie i usuwanie książek.

\subsection{Backlog Item 8}
\paragraph{Tytuł zadania.} Utworzenie podstrony modyfikującej, dodającej oraz usuwającej użytkowników zawartych w bazie danych.
\paragraph{Opis zadania.} Zadanie ma na celu stworzenie podstrony, pozwalającej na modyfikowanie bazy użytkowników.
\paragraph{Priorytet.} 3
\paragraph{Definition of Done.}Działające podstrony aplikacji umożliwiające dodawanie, modyfikowanie i usuwanie użytkowników.

\subsection{Backlog Item 9}
\paragraph{Tytuł zadania.} Utworzenie skryptów PHP: Sprawdzanie wypożyczonych książek przez klienta, prolongata wypożyczonych pozycji.
\paragraph{Opis zadania.} Zadanie ma na celu utworzenie skryptów w języku PHP pozwalających na podgląd danych wypożyczonych książek oraz prolongatę wybranych przez klienta pozycji.
\paragraph{Priorytet.} 3
\paragraph{Definition of Done.} Działające skrypty PHP wyświetlające wypożyczone książki oraz obsługujące funkcję prolongaty przez klienta.

\subsection{Backlog Item 10}
\paragraph{Tytuł zadania.} Utworzenie skryptów obługujących logowanie do systemu.
\paragraph{Opis zadania.} Zadanie ma na celu utworzenie skryptów pozwalających zalogować się zarówno klientowi jak i pracownikowi biblioteki do systemu obsługującego bibliotekę.
\paragraph{Priorytet.} 3
\paragraph{Definition of Done.} Działające skrypty umożliwiające logowanie do systemu.

\subsection{Backlog Item 11}
\paragraph{Tytuł zadania.}Poprawienie funkcjonalności opcji modyfikacji danych książki.
\paragraph{Opis zadania.}Zadanie ma na celu usprawnienie działania skryptu umożliwiającego modyfikacje danych książek.
\paragraph{Priorytet.}3
\paragraph{Definition of Done.}Podczas poprawiania danych poprzednie informacje zostaną wyświetlone w formularzu.

\subsection{Backlog Item 12}
\paragraph{Tytuł zadania.}Dodanie funkcjonalności obsługi błędów przy wprowadzaniu informacji do bazy danych.
\paragraph{Opis zadania.}Zadanie ma na celu dodanie aktualizacji skryptu dotyczącego funkcji obsługi błędów.
\paragraph{Priorytet.}3
\paragraph{Definition of Done.}Podczas próby dodania wpisu nie zawierającego danych użytkownik otrzyma informację o błędzie, wpis nie zostanie dodany do bazy danych.

\subsection{Backlog Item 13}
\paragraph{Tytuł zadania.} Wykonanie logo biblioteki.
\paragraph{Opis zadania.} Zadanie ma na celu stworzenie projektu graficznego logo dla biblioteki w programie Adobe Illustrator.
\paragraph{Priorytet.} 4
\paragraph{Definition of Done.} Logo biblioteki zawierające motyw książki wpleciony w nazwę aplikacji.

\subsection{Backlog Item 14}
\paragraph{Tytuł zadania.} Utworzenie podstrony wyświetlającej katalog książek.
\paragraph{Opis zadania.} Zadanie ma na celu stworzenie podstrony umożliwiającej podgląd listy książek.
\paragraph{Priorytet.} 5
\paragraph{Definition of Done.} Działająca strona wyświetlająca listę książek dla pracowników i czytelników.

\section{Sprint 1}
\subsection{Cel} Stworzenie pierwszej wersji produktu, która umożliwia dodawanie, usuwanie, modyfikowanie wpisów książek.
\subsection{Sprint Planning/Backlog}

\paragraph{Tytuł zadania.}Zaprojektowanie bazy danych.
\begin{itemize}
\item Estymata: L.
\end{itemize}

\paragraph{Tytuł zadania.}Utworzenie bazy danych.
\begin{itemize}
\item Estymata: L.
\end{itemize}

\paragraph{Tytuł zadania.}Utworzenie szablonu interfejsu webowego.
\begin{itemize}
\item Estymata: XL.
\end{itemize}

\paragraph{Tytuł zadania.}Utworzenie podstron: dodającej, modyfikującej jak i usuwającej wpisy książkowe w bazie danych.
\begin{itemize}
\item Estymata: L.
\end{itemize}

\paragraph{Tytuł zadania.} Utworzenie skryptów PHP: Modyfikacja, dodawanie i usuwanie książek.
\begin{itemize}
\item Estymata: L.
\end{itemize}

\paragraph{Tytuł zadania.} Wykonanie logo biblioteki.
\begin{itemize}
\item Estymata: M.
\end{itemize}

\subsection{Realizacja}

\paragraph{Tytuł zadania.} Zaprojektowanie bazy danych.
\subparagraph{Wykonawca.} Mateusz Zygmunt.
\subparagraph{Realizacja.} Utworzenie wstępnego projektu bazy za pomocą programu MS Access, zawierającego tabele oraz relacje. Schemat bazy danych na podstawie raportu relacji wyeksportowanego do pliku formatu pdf. Czas realizacji schematu był zgodny z estymatą, brak trudności podczas wykonywania zadania.

\paragraph{Tytuł zadania.} Utworzenie bazy danych.
\subparagraph{Wykonawca.} Łukasz Wojtas.
\subparagraph{Realizacja.} Baza została utworzona zgodnie z graficznym projektem bazy. Zawiera tabele przechowujące dane kont pracowników, czytelników biblioteki oraz książek które biblioteka posiada. W bazie istnieją również tabele pozwalające na zarządzanie wypożyczeniami książek. Baza i jej funkcje będzie rozbudowywana razem z kolejnymi wersjami aplikacji. Czas wykonania bazy oraz skryptu który ją tworzy był zgodny z estymatą. Mała trudność z przypisaniem powiązań między poszczególnymi tabelami została rozwiązana szybko, przy pomocy materiałów dotyczących języka SQL dostępnych w internecie.
 \begin{verbatim}
create database biblio character set utf8 collate utf8_unicode_ci;
use biblio;

CREATE TABLE czytelnicy (
id_czytelnika INT NOT NULL auto_increment,
imie VARCHAR(30),
login VARCHAR(30)NOT NULL,
haslo VARCHAR(60)NOT NULL,
PRIMARY KEY(id_czytelnika)
) ENGINE = InnoDB; 

CREATE TABLE pracownicy (
id_pracownika INT NOT NULL auto_increment,
imie VARCHAR(30),
login VARCHAR(30) NOT NULL,
haslo VARCHAR(60) NOT NULL,
stanowisko VARCHAR(30),
PRIMARY KEY(id_pracownika)
) ENGINE = InnoDB; 

CREATE TABLE gatunek (
id_gatunku INT NOT NULL auto_increment,
nazwa VARCHAR(30),
PRIMARY KEY(id_gatunku)
) ENGINE = InnoDB; 

CREATE TABLE ksiazki (
id_ksiazki INT NOT NULL auto_increment,
tytul VARCHAR(50) NOT NULL,
imie VARCHAR(30),
nazwisko VARCHAR(30),
wydawnictwo VARCHAR(50),
rok INT(10),
gatunek INT(10)
PRIMARY KEY(id_ksiazki),
FOREIGN KEY(gatunek) REFERENCES gatunek(id_gatunku)
) ENGINE = InnoDB; 


CREATE TABLE wypozyczenia (
id_wypozyczenia INT NOT NULL auto_increment,
id_czytelnika VARCHAR(30) NOT NULL,
id_ksiazki VARCHAR(30) NOT NULL,
data_wypozyczenia DATE NOT NULL,
data_zwrotu DATE NOT NULL,
stan_wypozyczenia enum('0','1') NOT NULL DEFAULT '0',
PRIMARY KEY(id_wypozyczenia),
CONSTRAINT c_fkw1 FOREIGN KEY(id_czytelnika) REFERENCES czytelnicy (id_czytelnika)
ON UPDATE CASCADE ON DELETE CASCADE,
CONSTRAINT c_fkw2 FOREIGN KEY(id_ksiazki) REFERENCES ksiazki (id_ksiazki)
ON UPDATE CASCADE ON DELETE CASCADE,
) ENGINE = InnoDB; 




CREATE OR REPLACE VIEW ksiazka_gatunek(id_ksiazki, tytul, imie, nazwisko, wydawnictwo, rok, gatunek)
 AS SELECT id_ksiazki, tytul, imie, nazwisko, wydawnictwo, rok, gatunek.nazwa FROM ksiazki, gatunek WHERE ksiazki.gatunek=gatunek.id_gatunku;
CREATE OR REPLACE VIEW wypozyczone (id_wypozyczenia, tytul, login, data_wypozyczenia, data_zwrotu, stan_wypozyczenia) 
AS SELECT id_wypozyczenia, ksiazki.tytul, czytelnicy.login, data_wypozyczenia, data_zwrotu, stan_wypozyczenia 
FROM wypozyczenia, czytelnicy, ksiazki where ksiazki.id_ksiazki=wypozyczenia.id_ksiazki 
and czytelnicy.id_czytelnika=wypozyczenia.id_czytelnika;

\end{verbatim}>>.


\paragraph{Tytuł zadania.}Utworzenie szablonu interfejsu webowego.
\subparagraph{Wykonawca.} Karolina Pasiut.
\subparagraph{Realizacja.} Szablon strony wykonany przy użyciu języków HTML oraz CSS. Czas realizacji strony jest zgodny z estymatą. 

\paragraph{Tytuł zadania.} Utworzenie podstron: dodającej, modyfikującej jak i usuwającej wpisy książkowe w bazie danych.
\subparagraph{Wykonawca.} Grzegorz Wygoda.
\subparagraph{Realizacja.}Podstrony zostały wykonane w języku HTML oraz CSS. Trudności sprawiło ustawienie podstron w prawidłowej pozycji względem siebie, zostały one rozwiązane poprzez użycie komendy div. Zadanie zostało wykonane w czasie wyznaczonym estymatą, oraz zawiera ustalone podstrony. 
 \begin{verbatim}
<HTML><html>
<head>
<meta charset="utf-8">
<title>PANEL ADMINISTRATORA</title>
<link rel="stylesheet" href="biblio.css">
</head>
<body bgcolor="b3b3b3">
<div id="footer"></div> 
<div class="container">
 <header>
PANEL ADMINISTRATORA
</header>
</body>
</html>

<HTML><html>
<head>
<meta charset="utf-8">
<title>GODZINY OTWARCIA</title>
<link rel="stylesheet" href="biblio.css">
</head>
<body bgcolor="b3b3b3">
<div id="footer"></div> 
<div class="container">
  <header>
</body>
</html>

<HTML><html>
<head>
<meta charset="utf-8">
<title>JAK DOJECHAĆ</title>
<link rel="stylesheet" href="biblio.css">
</head>
<body bgcolor="b3b3b3">
<div id="footer"></div> 
<div class="container">
<header>
</body>
</html>

Stworzony plik css (fragment).

div.container {
    width: 100%;

    }

header{
    padding: 1em;
    color: white;
    background-color: #FF6347;		
    clear: left;
    text-align: center;
}

footer {
   position: absolute;
   color: white;
   bottom: 0;
   padding: 1em;
   text-align: center;
   width: 97%;
   height: 20px;   
   background:#FF6347;
}
\end{verbatim}>>.

\paragraph{Tytuł zadania.}  Utworzenie skryptów PHP: Modyfikacja, dodawanie i usuwanie książek.
\subparagraph{Wykonawca.} Łukasz Wojtas.
\subparagraph{Realizacja.} Stworzenie skryptów w języku PHP. Przypisanie kodu do utworzonej bazy danych, oraz powiązanie z podstronami (HTML). Zadanie wykonane zgodnie z ustaloną estymatą. Trudności z zapisem kodu PHP, sporadyczne błędy składni, problem z połączeniem skryptu php z bazą danych, problemy zostały rozwiązane.
Kod programu 
\begin{verbatim}
<?php
ob_start();
session_start();
if ($_SESSION['kto'] !== 'pracownik'){
	header('location: index.php');
}
?>
<HTML><html>
<head>
<meta charset="utf-8">
<title>Aplikacja COMFORT</title>
<link rel="stylesheet" href="biblio.css">
</head>
<body bgcolor="b3b3b3">
<div align="right" style="font-size: 16px">Jesteś zalogowany jako:<b><?php echo $_SESSION['imie']; ?></b></div>
<div align="right"><a href="wyloguj.php" margin="right 0px">Wyloguj</a></div>
<div id="footer"></div> 
<div class="container">
  

<header>
Zarządzanie czytelnikiem
</body>
</html>
\end{verbatim}>>.

\paragraph{Tytuł zadania.} Wykonanie logo biblioteki.
\subparagraph{Wykonawca.} Patrycja Pieniążek.
\subparagraph{Realizacja.}Stworzenie logo strony biblioteki za pomocą programu  Adobe Illustrator. Zadanie zostało wykonane zgodnie z estymatą, nie sprawiło większych trudności.

\subsection{Sprint Review/Demo}
Wyznaczony cel sprintu nie został w pełni osiągnięty oraz nie wszystkie zaplanowane zadania zostały zrealizowane. Zadaniem które nie zostało w pełni wykonane było utworzenie skryptów do modyfikacji i usuwania wpisów książkowych. W kolejnych sprintach należy uwzględnić dodanie funkcji obsługi błędów oraz usprawnienie korekty danych poprzez aplikację. Demonstracja produktu została zakłócona przez defekt oprogramowania na urządzeniu do niej przeznaczonym. Problem został rozwiązany poprzez szybką interwencję zespołu deweloperskiego i przeniesienie aplikacji ze wszystkimi niezbędnymi jej częściami na inne urządzenie. Po uporaniu się z problemem prezentacja produktu przebiegła sprawnie.

\section{Sprint 2}

\subsection{Cel} Stworzenie drugiej wersji produktu. Poszerzenie funkcjonalności aplikacji o możliwość dodawania, usuwania oraz modyfikacji danych użytkowników biblioteki.

\subsection{Sprint Planning/Backlog}

\paragraph{Tytuł zadania.} Utworzenie podstrony modyfikującej, dodającej oraz usuwającej użytkowników zawartych w bazie danych.
\begin{itemize}
\item Estymata: L.
\end{itemize}

\paragraph{Tytuł zadania.} Utworzenie skryptów PHP: Modyfikacja, dodawanie i usuwanie użytkowników.
\begin{itemize}
\item Estymata: L.
\end{itemize}

\paragraph{Tytuł zadania.} Poprawienie funkcjonalności opcji modyfikacji danych książki.
\begin{itemize}
\item Estymata: L.
\end{itemize}

\paragraph{Tytuł zadania.} Dodanie funkcjonalności obsługi błędów przy wprowadzaniu informacji do bazy danych.
\begin{itemize}
\item Estymata: L.
\end{itemize}

\paragraph{Tytuł zadania.} Rozbudowa bazy danych na potrzeby obsługi kolejnych funkcji aplikacji.
\begin{itemize}
\item Estymata: L.
\end{itemize}

\paragraph{Tytuł zadania.} <<Tytuł>>.
\begin{itemize}
\item Estymata: <<szacowana czasochłonność (w ,,koszulkach'')>>.
\end{itemize}
\subsection{Realizacja}


\paragraph{Tytuł zadania.}  Utworzenie podstrony modyfikującej, dodającej oraz usuwającej użytkowników zawartych w bazie danych.
\subparagraph{Wykonawca.} Patrycja Pieniążek 
\subparagraph{Realizacja.} <<Sprawozdanie z realizacji zadania (w tym ocena zgodności z estymatą).



\paragraph{Tytuł zadania.}  Utworzenie skryptów PHP: Modyfikacja, dodawanie i usuwanie użytkowników..
\subparagraph{Wykonawca.} Łukasz Wojtas
\subparagraph{Realizacja.} <<Sprawozdanie z realizacji zadania (w tym ocena zgodności z estymatą).



\paragraph{Tytuł zadania.} Poprawienie funkcjonalności opcji modyfikacji danych książki.
\subparagraph{Wykonawca.} Karolina Pasiut
\subparagraph{Realizacja.} <<Sprawozdanie z realizacji zadania (w tym ocena zgodności z estymatą).



\paragraph{Tytuł zadania.} Dodanie funkcjonalności obsługi błędów przy wprowadzaniu informacji do bazy danych.
\subparagraph{Wykonawca.} Mateusz Zygmunt
\subparagraph{Realizacja.} <<Sprawozdanie z realizacji zadania (w tym ocena zgodności z estymatą).



\paragraph{Tytuł zadania.}  Rozbudowa bazy danych na potrzeby obsługi kolejnych funkcji aplikacji.
\subparagraph{Wykonawca.} Grzegorz Wygoda
\subparagraph{Realizacja.} <<Sprawozdanie z realizacji zadania (w tym ocena zgodności z estymatą).



\paragraph{Tytuł zadania.} <<Tytuł>>.
\subparagraph{Wykonawca.} <<Wykonawca>>.
\subparagraph{Realizacja.} <<Sprawozdanie z realizacji zadania (w tym ocena zgodności z estymatą).


\subsection{Sprint Review/Demo}
<<Sprawozdanie z przeglądu Sprint'u -- czy założony cel (przyrost) został osiągnięty oraz czy wszystkie zaplanowane Backlog Item'y zostały zrealizowane? Demostracja przyrostu produktu>>.

\section*{<<Tutaj dodawać kolejne Sprint'y>>}


\begin{thebibliography}{9}

\bibitem{Cov} S. R. Covey, {\em 7 nawyków skutecznego działania}, Rebis, Poznań, 2007.

\bibitem{Oet} Tobias Oetiker i wsp., Nie za krótkie wprowadzenie do systemu \LaTeX  \ $2_\varepsilon$, \url{ftp://ftp.gust.org.pl/TeX/info/lshort/polish/lshort2e.pdf}

\bibitem{SchSut} K. Schwaber, J. Sutherland, {\em Scrum Guide}, \url{http://www.scrumguides.org/}, 2016.

\bibitem{apr} \url{https://agilepainrelief.com/notesfromatooluser/tag/scrum-by-example}

\bibitem{us} \url{https://www.tutorialspoint.com/scrum/scrum_user_stories.htm}

\end{thebibliography}

\end{document}

